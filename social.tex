\chapter{Fabrique social de la métropole}


\section{«~J'ai besoin de ma voiture~»}

«~J'ai besoin de ma voiture~» est une phrase fréquemment invoquée pour
justifier notre dépendance à l'automobile. Bien souvent, cette
affirmation reflète une réalité incontestable. Toutefois, elle nous
livre également un précieux indicateur que la métropole, les
départements et la région se doivent de prendre en considération. Il
est impératif de comprendre les contraintes qui rendent difficile le
passage à d'autres modes de transport et d'agir concrètement pour
faciliter cette transition.Il est tout aussi primordial que la
métropole communique de manière transparente et régulière sur les
initiatives mises en place et sur leur taux d'adoption. La réussite de
ces projets réside en grande partie dans la mobilisation et l'adhésion
des citoyens, à condition que ces projets soient clairement présentés
et justifiés.


\section{Le rôle singulier des écoles et des élèves}

Les enfants d'aujourd'hui façonnent le monde de demain. Leurs choix et
comportements influencent également ceux de leurs parents. Il est donc
primordial d'intégrer l'école, premier lieu de socialisation, dans nos
réflexions sur la mobilité de demain.

L'implication des jeunes ne doit pas être uniquement passive.  Nous
proposons de les engager dans l'évolution de leur propre mobilité, en
leur demandant de réfléchir et d'effectuer les changements.
Accompagnés par leur enseignants, nous suggérons que les écoliers
créent eux-même des jeux ou compétitions amiables (la
``gamification'') afin de s'encourager entre eux d'améliorer leur
propre mobilité.  Évidemment ils peuvent également réfléchir aux
incitations qui leur motiveraient les mieux.

Nous proposons en outre que les établissements scolaires publient
régulièrement en open data les modes de transport adoptés par leurs
élèves pour se rendre à l'école. Cela permettra de suivre et
d'encourager la proportion d'élèves optant pour des moyens de
déplacement écologiques.  Il ne s'agit pas d'instaurer un système de
surveillance lourd ni de mettre en place des pénalités, mais plutôt
d'établir un moyen transparent de suivre et encourager l'évolution des
habitudes de déplacement.

Bien que la métropole, le département et la région puissent définir
les grandes lignes des données à collecter, la mise en œuvre pourrait
être confiée aux acteurs locaux, y compris aux élèves eux-mêmes. Ces
derniers pourraient par exemple signaler leur mode de transport à leur
arrivée à l'école via une application mobile. Ou, dans un contexte
plus traditionnel, un enseignant pourrait simplement comptabiliser à
main levée les différents modes de transport. L'idée d'impliquer les
élèves dans la création d'un tel système, avec le soutien des
enseignants et administrateurs, est également une merveilleuse
opportunité pédagogique. L'essentiel est d'engager, de mesurer,
d'afficher et d'encourager une mobilité durable.

La publication régulière de ces données facilitera le dialogue entre
les différents acteurs – écoles, parents, élèves et élus. Ensemble,
ils pourront identifier des solutions pour encourager les modes de
déplacement non motorisés dans leurs quartiers.

Évidemment, un tel projet nécessiterait d'autres éléments : des
formations pour les enseignants sur la mobilité durable et les
avantages environnementaux et sociaux des modes de transport
écologiques.  Il serait judicieux également de sensibiliser les
parents pour qu'ils soutiennent leurs enfants dans ces choix.  Si les
écoles n'ont pas le matériel ou les infrastructures pour un transfert
modal des écoliers (abris et attaches vélo, par exemple, mais aussi
trottoirs et infrastructure cyclable autour des écoles), la métropole
doit être en mesure de participer à trouver des solutions rapides et
parfois tactiques, car un délai même de plusieurs années correspond à
un ``non'' absolu aux écoliers concernés.

Enfin, en intégrant les jeunes dans cette démarche, nous contribuerons
à changer leur perception des modes de transport. Il est crucial que
la nouvelle génération voie le choix d'alternatives écologiques non
pas comme une contrainte, mais comme une responsabilité et un
privilège. Trop souvent, la voiture est perçue comme un symbole de
maturité et d'indépendance. Il est temps de redéfinir ces aspirations
pour un avenir plus vert.
