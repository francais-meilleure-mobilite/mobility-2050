\chapter{Principes}

\section{La transition écologique et la mobilité}

La transition écologique et énergétique, souvent abordée de manière
continue et sans limite temporelle précise, nécessite un cadre de
réflexion sur les 25 prochaines années. Cette période nous offre le
temps nécessaire pour concevoir et achever cette transformation.

Les transports représentent environ un tiers des émissions de gaz à
effet de serre en France, une proportion élevée, surtout en
comparaison avec des pays moins dépendants du nucléaire, où cette part
est de 1/4. La transition écologique dans la mobilité revêt donc une
importance cruciale, notamment pour des régions dynamiques comme les
Pays de la Loire et des villes en effervescence comme Nantes.

Cette transition n'est pas seulement technique, mais aussi
sociétale. Elle exige des changements structurels et des ajustements
dans nos comportements individuels et collectifs. Avec une vision
claire, un engagement soutenu et une collaboration multisectorielle,
Nantes Métropole et les Pays de la Loire peuvent devenir des pionniers
de cette transformation essentielle pour notre avenir.


\section{La transparence : un impératif}

La transformation de la mobilité est une démarche sociétale qui
requiert compréhension et adhésion publiques. Pour garantir le succès
des projets à Nantes et dans les Pays de la Loire au cours des
prochaines décennies, la transparence doit être notre ligne
directrice.

\begin{enumerate}
\item L'annonce des objectifs : Pour instaurer la confiance, il est
  primordial d'annoncer clairement les objectifs visés. Ces derniers
  doivent refléter les aspirations de la collectivité tout en
  proposant des solutions tangibles aux problématiques de mobilité.
\item L'engagement du public : L'adhésion du public est cruciale. Il
  s'agit de faire des citoyens non seulement des bénéficiaires, mais
  aussi des acteurs de cette transformation. Les consultations
  publiques, les ateliers de réflexion et autres formes de
  participation citoyenne doivent être encouragés.
\item Suivi des avancements et transparence sur les retards : Tout
  projet est sujet à des aléas, et des retards peuvent
  survenir. Plutôt que de les cacher, il est essentiel de les
  communiquer ouvertement et d'expliquer leurs causes, tout en
  rassurant sur les mesures prises pour rectifier le tir.
\item Débat continu sur les principes de base : Les conditions
  changent, les technologies évoluent et les besoins de la population
  peuvent se modifier. Il est donc impératif de maintenir un débat
  continu sur les principes qui guident la transformation de la
  mobilité pour assurer un alignement constant avec les réalités du
  terrain.
\item Communication des plans d'action : Aucun projet ne survit intact
  à son premier contact avec le marché. C'est pourquoi il est
  important de discuter régulièrement des plans d'action, de les
  adapter si nécessaire, et de les communiquer de manière
  transparente.
\item L’importance de résultats partiels : La mobilité de 2050
  commence aujourd'hui. Il est donc crucial de valoriser les étapes
  intermédiaires et de célébrer les succès, même partiels. Ce n'est
  pas simplement une question de "livraison finale", mais de
  "livraisons continues". Chaque avancement, chaque progrès doit être
  communiqué de manière fiable et célébré comme une étape vers
  l'objectif ultime.
\end{enumerate}

La transparence n'est pas un luxe, mais une nécessité. Elle garantit
l'adhésion du public, renforce la confiance dans le projet et assure
une meilleure réactivité face aux défis rencontrés. Pour une mobilité
durable, efficiente et écologique à Nantes et dans les Pays de la
Loire, l'ouverture, la communication et l'engagement public sont des
piliers incontournables.


\section{La véritable rôle de l'innovation dans la mobilité}

L'innovation est au cœur de toute transformation majeure de la
société, et encore plus quand il s'agit d'un domaine aussi vital que
la mobilité. Cependant, son concept est bien souvent mécompris.

L'innovation n'est pas une fin en soi, mais un ensemble de techniques
pour atteindre un objectif précis. Elle s'amorce en s'inspirant de ce
qui a déjà fonctionné ailleurs dans des contextes similaires. Il est
essentiel de comprendre non seulement les méthodes employées, mais
surtout les raisons de leur succès. Au fur et à mesure que les équipes
maîtrisent les compétences du nouveau domaine, elles peuvent commencer
à expérimenter et à innover. C'est là que réside le véritable
caractère de l'innovation: tester une nouveauté à petite échelle,
s'appuyant sur une compréhension approfondie du domaine en question,
observer la réaction du public, puis formuler des améliorations. Ce
cycle d'expérimentation et d'amélioration se répète jusqu'à atteindre
l'excellence ou le niveau souhaité.

Il est tout aussi crucial de définir ce que l'innovation n'est
pas. Mettre en œuvre une idée puis passer à autre chose sans suivis
n'est pas de l'innovation: c'est un pari imprudent sur une hypothèse
non vérifiée. Ne pas observer, mesurer et rendre compte des résultats
équivaut à une abdication pure et simple de ses responsabilités.

Lorsque nous envisageons l'avenir de la mobilité à Nantes et dans les
Pays de la Loire, il est impératif de garder à l'esprit cette
définition authentique de l'innovation. Elle doit guider nos actions
et décisions pour construire une mobilité plus efficace, durable et
écologique pour tous.


\section{Vision holistique pour une mobilité durable}

Les projets esquissés dans ce livre blanc sont ambitieux, que ce soit
à titre individuel ou collectif. Aborder ces projets de manière
séquentielle serait une stratégie vouée à l’échec. Deux raisons
principales viennent appuyer cette affirmation :

Premièrement, la majorité de ces propositions nécessitera de
nombreuses années de planification, de concertation, de mise en œuvre,
de débats, d’améliorations et d’accompagnement jusqu’à leur
aboutissement satisfaisant. Il est donc irréaliste de croire qu'une
approche linéaire serait la plus efficiente.

Deuxièmement, la réussite de ces projets repose fortement sur
l'enthousiasme de la communauté. Cet élan collectif naît d'une vision
d'ensemble et d'un engagement commun envers un objectif ultime : une
mobilité respectueuse de l’humain, du budget et de l’environnement à
l’horizon d’une génération.

Il est donc essentiel d’avancer simultanément sur l'ensemble de ces
propositions. Cette démarche a pour but d’encourager la population à
envisager la mobilité souhaitée pour leurs enfants et pour les
générations futures. Il s’agit aussi d’engager un débat qui dépasse
largement les propositions conventionnelles, souvent limitées à la
durée d’un mandat politique.

Pour mener à bien cette transformation, nous devons veiller à ne pas
tomber dans les pièges habituels d’une vision à court terme. C’est un
chantier de grande envergure qui attend Nantes et les Pays de la
Loire. C’est ensemble, avec une vision à long terme, que nous
réussirons à repenser notre mobilité.


\section{Le pouvoir du coup de pouce pour une mobilité durable}

Il est rare que l’on accueille favorablement une directive nous
exhortant à changer nos comportements fondamentaux. En général,
l'individu est enclin au changement lorsqu'il est spontané et non
imposé.

Dans le domaine de la mobilité, nous ne souhaitons même pas que chacun
modifie ses habitudes simultanément. Une telle mutation soudaine
surchargerait notre infrastructure et déstabiliserait notre
économie. En revanche, notre ambition est qu'un pourcentage minime,
mais constant voire augmentant, de la population reconsidère ses modes
de déplacement chaque année. C'est là la clé d'une transformation
radicale de la mobilité en l'espace de vingt-cinq ans.

D'abord, il est impératif de ne pas sommer les gens de changer, et
encore moins de les y contraindre. Notre approche privilégiée est de
les inciter à essayer de nouvelles alternatives, ne serait-ce que
ponctuellement, pour s’y familiariser. Nous pourrions ainsi proposer
des incitatifs financiers ou d'autres avantages attrayants.

Ensuite, et c'est tout aussi essentiel, nous devons encourager ceux
qui ne désirent pas changer à conserver leurs habitudes. Nous leur
demandons simplement d'accepter et de reconnaître les changements
adoptés par d'autres. À ceux pour qui la voiture est indissociable de
leurs trajets quotidiens, par exemple, nous pouvons arguer qu’il est
primordial que certains délaissent leur véhicule pour que la
transition soit efficace. Ainsi, nous proposons aux plus réticents que
d'autres « prennent le relais » pour eux.

Nos voisins anglo-saxons ont un terme pour décrire cette approche
subtile : le "nudging" ou "coup de pouce". Une campagne basée sur ce
concept rencontre nettement moins d'opposition qu'une initiative
coercitive, car elle ne présente guère de points de friction auxquels
résister.

Ce que nous proposons est une transformation douce, graduelle, mais
résolument tournée vers l'avenir, afin de garantir une mobilité plus
respectueuse de l'environnement et adaptée aux défis du XXIe siècle.


\section{De la fourniture de routes à la fourniture de mobilité}

En France, la mobilité a souvent été associée à l'infrastructure
routière, mettant en avant l'automobile et le conducteur comme
éléments centraux. Cependant, dans un contexte de préoccupation
croissante pour l'environnement et l'efficacité énergétique, il est
temps de repenser notre approche, passant ainsi de la construction de
routes à la facilitation de la mobilité.

\textbf{Implication environnementale.} Centrer nos infrastructures
autour des routes, c'est avant tout encourager une culture
profondément automobile, nuisant à l'environnement en imperméabilisant
les sols, réduisant les espaces verts et augmentant la congestion
routière. En outre, la perception accrue de danger pour les autres
usagers décourage la mise en place d'autres modes de transport.

\textbf{Implication sociale.} L'accent sur l'automobile crée des
inégalités, excluant les jeunes, les personnes âgées et les ménages à
faibles revenus qui ne peuvent pas posséder ou conduire une
voiture. Diversifier les options de mobilité favorise l'inclusion,
permettant à chacun de se déplacer librement et en sécurité, d'être
acteur de sa propre autonomie. C'est un levier puissant de cohésion
sociale.

\textbf{Implication économique.} Adopter une vision élargie de la
mobilité, c'est ouvrir la porte à de nouvelles dynamiques
économiques. Prenons l'exemple du vélo : son développement stimule les
entreprises locales. Un euro investi dans les mobilités actives et
douces profite majoritairement à l'économie locale, tandis que le même
euro dépensé pour l'automobile est souvent transféré à de grandes
multinationales pétrolières et automobiles, parfois même hors de nos
frontières. Par ailleurs, une cité qui respire, dépourvue de
congestion permanente, est naturellement plus attractive pour les
investisseurs et les visiteurs, favorisant ainsi le dynamisme
économique.

Les pouvoirs publics doivent revoir leur conception de la mobilité, en
faisant de la voiture une option parmi d'autres. Que ce soit à Nantes
ou dans les Pays de la Loire, intégrer les dimensions
environnementale, sociale et économique, est impératif. En repensant
notre approche de la mobilité, nous posons les bases de villes plus
agréables à vivre et de sociétés plus équilibrées et inclusives.


