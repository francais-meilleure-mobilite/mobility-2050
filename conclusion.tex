\chapter{Conclusions}

Dans ce livre blanc, nous avons détaillé les ambitions et projets
visant à transformer en profondeur la mobilité à Nantes Métropole et
dans les Pays de la Loire durant les 25 prochaines années. Cette durée
représente une opportunité majeure pour orienter, construire et guider
nos concitoyens vers une transition durable. L'objectif est clair :
établir un système de transport à la fois rentable, efficace et
respectueux de l'environnement. Cette mission s'ancre autour de
principes clés : transparence, clarté des objectifs et suivi des
avancées, tout en mettant en lumière les défis rencontrés.

Un premier axe d'action met l'accent sur la marche et le vélo. Cela
implique une refonte du stationnement, tant automobile que vélo, et
une révision des limites de vitesse. À moyen terme, éliminer les
obstacles verticaux devient impératif pour faciliter la mobilité de
tous, notamment les seniors, parents avec jeunes enfants et personnes
à mobilité réduite.

Le deuxième axe se concentre sur les transports en commun. Parmi les
solutions envisagées : intensifier la fréquence et l'amplitude des
services, remettre en service d'anciennes lignes et envisager des
services de cars express. Face à la congestion à la gare centrale de
Nantes, nous suggérons des mesures simples pour améliorer la fluidité
et la capacité.  De surcroît, nous soulignons l'importance d'une
meilleure accessibilité aux trains et cars. Le transport de vélos ne
doit plus être considéré comme une nécessité marginale réservée aux
sportifs, et l'accessibilité pour tous demeure une priorité.

Nous avons également souligné la nécessité de repenser
l'infrastructure de la métropole et de la région. Cela englobe la
réduction de la présence de véhicules lourds, l'adaptation de nos
véhicules de service à des rues conçues à l'échelle humaine, la
révision des obligations relatives à la construction de parkings, et
surtout, la priorisation des modes de transports actifs et doux dans
la conception urbaine, éliminant ainsi l'argument récurrent du "manque
d'espace".

Enfin, il est essentiel de tenir compte des signaux envoyés par nos
concitoyens quant à la nécessité de leurs voitures. La collaboration
avec les écoles et l'engagement des élèves sont cruciaux pour induire
une transformation culturelle durable.


\section*{À propos des Mobilitains}

Les Mobilitains, basées à Nantes Métropole, militent pour des
améliorations de la mobilité en région nantaise et dans les Pays de la
Loire.


\section*{Droit d'auteur}

Vous êtes autorisé à

\begin{itemize}
\item \textbf{Partager :} copier, distribuer et communiquer le
  matériel par tous moyens et sous tous formats.
\item \textbf{Adapter :}  remixer, transformer et créer à partir du
  matériel pour toute utilisation, y compris commerciale.
\end{itemize}

selon les conditions suivantes :

\begin{itemize}
\item \textbf{Attribution :} Vous devez créditer l'Œuvre, intégrer un
  lien vers la licence et indiquer si des modifications ont été
  effectuées à l'Oeuvre. Vous devez indiquer ces informations par tous
  les moyens raisonnables, sans toutefois suggérer que l'Offrant vous
  soutient ou soutient la façon dont vous avez utilisé son Oeuvre.
\item \textbf{Pas de restrictions complémentaires :} Vous n'êtes pas
  autorisé à appliquer des conditions légales ou des mesures
  techniques qui restreindraient légalement autrui à utiliser l'Oeuvre
  dans les conditions décrites par la licence.
\end{itemize}

\bigskip
{\small
\copyright{} Les Mobilitains,
CC BY-SA 4.0 \\
\url{https://creativecommons.org/licenses/by-sa/4.0/deed.fr}
}