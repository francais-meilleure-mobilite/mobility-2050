\chapter{Structure de la métropole}

Pour rendre nos villes plus vivables, fluides et écologiques, il ne
suffit pas seulement de repenser nos modes de transport, mais aussi de
revoir la structure même de nos espaces urbains. Dans ce chapitre,
nous aborderons trois aspects essentiels de l'infrastructure
métropolitaine : la conception des carrefours, la gestion du
stationnement et la maintenance des voies.


\section{Poids lourds et carrefours}

Les poids lourds jouent un rôle central dans notre tissu économique,
garantissant la livraison de marchandises et le réapprovisionnement de
nos magasins. Toutefois, leur présence omniprésente façonne également
notre réseau routier, avec des aménagements tels que des voies larges
et des rayons de braquage étendus. Malheureusement, cette
infrastructure, bien qu'elle soit adaptée aux besoins des véhicules
lourds, peut compromettre la sécurité des piétons et des cyclistes.

En effet, la cohabitation sur nos rues entre ces mastodontes roulants
et les cyclistes ou piétons expose ces derniers à des risques
accrus. Alors, comment pouvons-nous protéger les usagers de la route
les plus vulnérables tout en assurant un transport efficace des
marchandises?

Il est essentiel d'envisager, sur le long terme, la réduction
progressive du recours aux poids lourds en milieu urbain. Pour ce
faire, repenser l'aménagement de nos rues est une première étape
cruciale. Dans ce contexte, l'adoption de solutions telles que les
palettes autonomes et les vélos cargo (pour les charges plus petites)
se présente comme une alternative pertinente. Non seulement ces
options offrent un moyen de transport de marchandises plus respectueux
de l'environnement, mais elles contribuent également à une meilleure
fluidité du trafic et à la diminution des congestions.

En adoptant une vision novatrice et en plaçant la sécurité et la
durabilité au cœur de nos politiques de transport, nous pouvons
transformer nos villes et nos régions en espaces plus verts, plus
sécurisés et plus agréables pour tous.

\refs

\begin{itemize}
\item \link{https://www.supplychaininfo.eu/vehicule-livraison-autonome/}
\item \link{https://www.groupestarservice.com/blog/vehicule-livraison-autonome/}
\item \link{https://www.zieglergroup.com/post/fr/pourquoi-le-groupe-ziegler-introduit-des-vehicules-de-livraison-autonomes-en-europe/}
\item \link{https://www.autoplus.fr/environnement/nuro-vehicule-autonome-livraison-552959.html\#item=1}
\item \link{https://www.usine-digitale.fr/article/nuro-presente-sa-troisieme-generation-de-robot-autonome-de-livraison.N1175542}
\item \link{https://adameo.com/blog/vehicule-autonome-et-transport-routier/}
\item \link{https://www.paris.fr/pages/logistique-marchandises-livraisons-4738}
\item \link{https://cdn.paris.fr/paris/2019/07/24/4cdaf0a5ab115898137e9093da69ae36.pdf}
\item \link{https://cdn.paris.fr/paris/2022/10/03/1bddb96a70a5b8df92b3258433fc83c3.pdf}
\item \link{https://www.paris.fr/pages/comment-paris-veut-repenser-sa-logistique-urbaine-21381}
\end{itemize}


\section{Services urbains}

Au fil du temps, nos villes se sont adaptées à accueillir les
véhicules lourds, ouvrant la voie aux services essentiels tels que les
pompiers et la collecte des déchets qui se servent presque uniquement
avec des véhicules lourds. Cependant, ces adaptations ont également
créé des barrières aux modes de transport plus écologiques et centrés
sur l'humain, comme la marche et le cyclisme. Alors que nous
envisageons un avenir où les rues sont plus accueillantes pour les
humains, nous devons également repenser la manière dont ces services
essentiels sont fournis.

Nous pouvons réduire la dépendance aux poids lourds tout en maintenant
des services essentiels efficaces et accessibles.  Il s'agit de penser
au remplacement avec du matériel plus adapté au fur et à mesure que
nous mettons à jour naturellement les véhicules de service actuels.

Nous offrons quelques exemples :

\textbf{Freiburg, Allemagne: Le modèle des "Vauban".}
Le quartier Vauban à Fribourg est une référence en matière de
conception urbaine écologique. Ici, la plupart des rues sont conçues
comme des "zones de rencontre", où les piétons et les cyclistes ont la
priorité. Pour résoudre la question des services d'urgence et de la
collecte des déchets, la ville a introduit des points de collecte
centralisés pour les déchets, éliminant ainsi le besoin de camions
poubelles pour naviguer dans de petites rues. Les services d'urgence
utilisent également des véhicules plus petits et plus agiles adaptés
aux rues étroites du quartier.

\textbf{Barcelone, Espagne: superblocks (superilles).}
Barcelone a commencé à mettre en œuvre le concept de "superblocks", de
grandes zones où la circulation est limitée et où les espaces sont
principalement réservés aux piétons et aux cyclistes. Pour la
prestation de services, des itinéraires spécifiques sont conçus pour
que les véhicules de service puissent accéder aux bâtiments sans
perturber la tranquillité des zones piétonnières. Ces routes spéciales
sont utilisées pour les services essentiels comme les pompiers et la
collecte des déchets.

\textbf{Ghent, Belgique: plan de circulation.}
Gand a mis en place un plan audacieux qui divise la ville en six zones
distinctes, accessibles uniquement par des routes périphériques. À
l'intérieur de ces zones, la priorité est donnée aux piétons, aux
cyclistes et aux transports en commun. Pour les services municipaux,
des véhicules plus compacts et écoénergétiques ont été adoptés. De
plus, les services de collecte de déchets encouragent le tri sélectif,
ce qui réduit le volume et la fréquence des ramassages.


\section{Abandonner l'exigence de création de places de parking}

Dans cette section, nous interrogeons l'obligation historique de créer
des places de stationnement voiture au sein des communes. Dans une
optique de mobilité durable à Nantes et dans les Pays de la Loire sur
les 25 années à venir, il est impératif de reconsidérer cette
pratique.

L'existence de places de stationnement voiture stimule l'usage de la
voiture individuelle. En offrant un espace privé pour le
stationnement, on incite à posséder une voiture. Ceci va à l'encontre
de nos ambitions de privilégier des moyens de transport
éco-responsables tels que les transports en commun, la marche ou le
vélo.

De plus, intégrer des places de stationnement dans les projets
immobiliers alourdit les coûts. Un parking souterrain peut coûter
entre 30 000 € à 50 000 € par emplacement, augmentant ainsi le coût
des logements. Ces frais viennent s'ajouter au prix de revient des
nouveaux logements, rendant ainsi l'accession à la propriété plus
onéreuse pour les citoyens.

Dans une perspective similaire, il convient d'encourager le citoyen
désireux de transformer de manière permanente son emplacement de
stationnement en un espace d'habitation, et ce, sans imposition de
frais.

Il est essentiel de repenser la systématisation des places de
stationnement dans les nouveaux projets urbains. Favorisons les
infrastructures qui réduisent la dépendance à la voiture. Cela
améliore la qualité de vie, diminue la congestion routière et préserve
l'environnement.

Il ne s'agit pas de proscrire totalement les places de stationnement,
mais de proscrire simplement l'obligation d'en créer. Les villes ayant
adopté cette approche ont constaté une réduction de près de 50~\% des
nouveaux emplacements créés.



\section{Repenser le stationnement commercial}

Pour une politique de transport durable, nous devons réviser notre
approche du stationnement commercial à Nantes et dans les Pays de la
Loire. Cette section traite des modifications nécessaires du code
municipal à ce sujet.

Au lieu de privilégier d'importants parkings en façade des magasins,
nous suggérons leur déplacement à l'arrière. Cette proposition
favorise l’expérience client, l'accessibilité et respecte
l'environnement. Repositionner les parkings transformera la perception
visuelle, mettant l'accent sur l'accès piétonnier et cycliste tout en
diminuant la dominance de l'automobile. Cela facilite également le
déplacement des clients entre commerces adjacents à pied. Le résultat
? Un commerce et un quartier qui valorisent les déplacements doux.

Par ailleurs, nous préconisons l'abolition de la contrainte de parking
pour les nouveaux commerces et ceux existants souhaitant
s'étendre. Cette liberté permettrait aux commerçants de réinventer
leurs espaces sans sacrifier de précieux mètres carrés pour le
stationnement voiture. Toutefois, nous ne sommes pas pour une
interdiction totale des parkings, mais nous croyons que c'est au
commerçant de définir le nombre de places adapté à ses besoins et à
son modèle économique.


\section{Travaux et restitution de la rue}

Lors de projets routiers entrepris par des entreprises privées, il est
primordial de veiller à une remise en état impeccable de la
chaussée. Cela assure non seulement la sécurité et le confort des
cyclistes, mais préserve également la longévité du revêtement. Les
chaussées mal restaurées se dégradent prématurément, engendrant des
coûts supplémentaires pour la métropole lors des nécessaires
interventions de resurfaçage.

La métropole se doit de garantir une supervision rigoureuse de ces
travaux. Elle pourrait également instaurer un système simplifié
permettant aux citoyens de signaler toute malfaçon sur la route. Cette
démarche collaborative, en impliquant les usagers, permettrait de
maintenir un niveau élevé de qualité de nos infrastructures routières.

Il s'agit là d'une question centrale pour la métropole. Au-delà du
confort des cyclistes, ces mesures sont également stratégiques du
point de vue financier. Un suivi rigoureux, appuyé par la vigilance
citoyenne, est essentiel pour une gestion durable et responsable de
notre réseau routier.

\section{Doux d’abord}

Lors de la conception de nouveaux espaces et quartiers urbains, un
changement radical de nos priorités en matière d'aménagement
s'impose. Notre proposition repose sur une approche claire et
résolument tournée vers l'avenir : mettre au premier plan la
conception des transports collectifs, suivis de près par les
déplacements piétonniers et cyclistes, et enfin, considérer
l'automobile.

Il ne s'agit pas de marginaliser l'automobile, mais de reconnaître une
réalité évidente : bien qu'essentielle, elle requiert une part
disproportionnée de notre espace urbain. Historiquement, nous avons
trop souvent adopté une stratégie axée sur la voiture, en reléguant
les transports collectifs, les infrastructures piétonnes et
éventuellement cyclables à une réflexion ultérieure, et souvent
marginale, justifiée par un prétendu manque d'espace.

Prenons l'exemple de l'Île de Nantes et du Quartier Mellinet : ces
zones, bien que récemment réaménagées, portent les marques d'une
planification priorisant excessivement l'automobile. L'espace,
idéalement destiné à promouvoir une mobilité douce et active, est
malheureusement dominé par des infrastructures orientées vers la
voiture. Si cette approche ne néglige pas totalement les espaces
verts, les zones piétonnes et les pistes cyclables sécurisées, elle
les relègue néanmoins à un second plan.

Pour envisager une mobilité durable, efficiente et respectueuse de
l'environnement à Nantes et dans les Pays de la Loire au cours des 25
prochaines années, nous devons adopter une vision rénovée de
l'aménagement urbain. C'est en redéfinissant nos priorités pour mettre
en avant les modes de transport éco-responsables et en concevant des
espaces urbains qui répondent aux aspirations modernes des citoyens
que nous parviendrons à une transformation profonde de notre mobilité.


\section{Réduction des îlots de chaleur urbains}

Les villes contemporaines, de par leur forte minéralisation,
contrastent avec les vastes étendues de prés et forêts qui les ont
précédées. Ce phénomène accentue la capacité des villes à capter la
radiation solaire et à la réémettre, jour comme nuit, créant ce que
l'on nomme des îlots de chaleur urbains (ICU). Ces ICU ne représentent
pas uniquement un enjeu de confort et de santé publique, mais
influencent également directement nos comportements de mobilité. En
effet, face à une température urbaine accablante, nombre de nos
concitoyens peuvent être tentés d'opter pour leurs véhicules
climatisés au détriment des transports collectifs, du vélo ou de la
marche.

Heureusement, la métropole dispose de plusieurs moyens économiques
pour combattre cet effet d'îlot de chaleur. Il est essentiel
d'atténuer l'absorption et l'émission de chaleur, tout en favorisant
les mécanismes naturels de rafraîchissement, tels que la circulation
de l'air et l'évapotranspiration.

Une première solution consiste à revisiter notre code d'urbanisme pour
encourager l'adoption de toitures réfléchissantes. Les modifications
apportées aux fenêtres pour réduire leur absorption thermique, ainsi
que l'intégration de panneaux solaires capables de transformer
l'énergie sans générer de chaleur, sont également des voies
prometteuses. Concernant les voies de circulation, typiquement
construites d'un mélange de goudron et de pierres concassées (macadam,
tarmac), l'utilisation de pierres plus claires pourrait réduire leur
capacité à retenir la chaleur.

Un autre axe d'intervention, d'égale importance, est la promotion de
la végétalisation urbaine. Les arbres, particulièrement efficaces pour
l'évapotranspiration, jouent un rôle central, mais même les petites
plantations urbaines contribuent au rafraîchissement de nos villes. Si
cette approche représente une solution économique, elle nécessite
cependant un engagement politique fort. En effet, l'espace dédié à
cette végétalisation se fera souvent au détriment de l'espace alloué à
la voiture. Cela souligne l'importance cruciale de l'ensemble des
mesures préconisées dans ce livre blanc pour une mobilité urbaine plus
durable et respectueuse de l'environnement.